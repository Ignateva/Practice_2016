\documentclass[a4paper,14pt]{report} %размер бумаги устанавливаем А4, шрифт 14пунктов
\usepackage [utf8x] {inputenc} %включаем свою кодировку
\usepackage [T2A] {fontenc}
\linespread{1.3} %полуторный межстрочный интервал
\usepackage[english,russian]{babel}%используем русский и английский языки с переносами
\usepackage[dvips]{graphicx} 
\graphicspath{{images/}}%путь к рисункам
\usepackage[left=3cm,right=2cm,top=2cm,bottom=2cm,bindingoffset=0cm]{geometry}
\renewcommand\contentsname{Projects List} %%% renaming the Table of Contents

\begin{document}
\begin{titlepage}

\begin{center}   
{\bfseries  
ФЕДЕРАЛЬНОЕ ГОСУДАРСТВЕННОЕ АВТОНОМНОЕ ОБРАЗОВАТЕЛЬНОЕУЧРЕЖДЕНИЕ ВЫСШЕГО ОБРАЗОВАНИЯ,САНКТ-ПЕТЕРБУРГСКИЙ НАЦИОНАЛЬНЫЙ ИССЛЕДОВАТЕЛЬСКИЙ УНИВЕРСИТЕТ ИНФОРМАЦИОННЫХ ТЕХНОЛОГИЙ,МЕХАНИКИ И ОПТИКИ
\par\smallskip
Факультет Компьютерных технологий и управления
\par\smallskip
Кафедра Вычислительной техники\\
}
\par\smallskip
\mbox{Направление подготовки (специальность): 09.03.04 - «Программная инженерия»}

\vspace{3cm} 
{\bfseries
ОТЧЕТ 
\par\medskip
по практике
}
\end{center}  

\par\noindent \smallskip
{\bfseries Тема задания:}
\par\noindent \smallskip
{\bfseries Студент: } Игнатьева Ю.В. {\bfseries Группа: } P3311
\par\noindent \smallskip
{\bfseries Руководитель практики: } Соснин Владимир Валерьевич
\par\noindent \smallskip
{\bfseries Оценка, рекомендуемая руководителем: \line(1,0){75} \quad \line(1,0){75}}
\vspace{1cm}

\parindent=8cm
{\bfseries Практика пройдена с оценкой : \line(1,0){85} \par \bigskip
Подписи членов комиссии \par \bigskip

\begin{flushright}  
\line(1,0){100}(\line(1,0){100}) \par \bigskip
\line(1,0){100}(\line(1,0){100}) \par \bigskip
\line(1,0){100}(\line(1,0){100}) \par \bigskip
\end{flushright} 

\parindent=8cm
Дата \line(1,0){100} }
\vspace{\fill}
\begin{center}  
\bfseries Санкт-Петербург \par 2016
\end{center} 

\end{titlepage}% это титульный лист
\tableofcontents % это оглавление, которое генерируется автоматически
\chapter{Система компьютерной вёрстки TeX (LaTeX)}
\section{Краткое описание LaTeX}
В 1978 году профессор Гарвардского университета Дональд Кнут опубликовал первую версию систему для верстки текстов, которая ныне известна как TeX.  Кнут писал TeX во времена появления цифрового печатающего оборудования, чтобы изучить его потенциал и, в особенности, обратить тенденцию ухудшения типографского качества, которую он наблюдал на примере его собственных книг. TeX произносится как "тех". Последняя буква X в названии TeX — вовсе не английская буква "икс" (x), а греческая "хи". TeX считается наиболее качественной системой подготовки текстов. Как сказано в словаре компьютерных терминов (Douglas Downing and Michael A. Covington. Dictionary of Computer and Internet Terms. 8th ed. Barron's Educational Series. New York: Woodbury, 2003.) TeX определяет стандарт, к которому пытаются приблизиться другие настольные издательские системы.
\par
В начале восьмидесятых годов XX века Лесли Лампорт разработал систему LaTeX, которая была основана на форматирующих средствах TeX'а. LaTeX - макропакет, который позволяет при помощи уже заранее определенных, профессиональных макетов верстать авторам их работы с высоким типографским качеством. LaTeX произносится как "латех". Чтобы пользоваться системой LaTeX и создавать удобные для чтения текстовые произведения, совсем не надо быть умелым пользователем системы ТеХ - достаточно выбрать готовый стиль и использовать несколько простых команд в зависимости от того, что нужно в данном случае. Набор макросов, для решения задач, пополнялся с каждым годом, поэтому в 1992 году был организован файловый архив CTAN. CTAN - это акроним "Comprehensive TeX Archive Network". Будучи распространяемым под лицензией LaTeX Project Public License, LaTeX относится к свободному программному обеспечению.
\par
Главная идея LaTeX состоит в том, что теперь пользователь может сконцентрировать свои усилия на содержании и структуре текста, о том, что он напишет, не заботясь о деталях его оформления. Готовя свой документ, автор указывает логическую структуру текста (разбивая его на главы, разделы, таблицы, изображения), а LaTeX решает вопросы его отображения. Так содержание отделяется от оформления. Оформление при этом или определяется заранее (стандартное), или разрабатывается для конкретного документа. В конце девяностых годов XX века такая идея отделения содержания от оформления нашла своё продолжение в XML - расширяемом языке разметки (eXtensible Markup Language).
\par
В России TeX и LaTex появились в конце восьмидесятых годов, тогда же был разработан алгоритм автоматического переноса русских слов. 
\par
Пакет позволяет автоматизировать многие задачи набора текста и подготовки статей, включая
\begin{enumerate} %перечень
\item набор текста на нескольких языках;
\item нумерацию разделов и формул;
\item перекрёстные ссылки;
\item размещение иллюстраций и таблиц на странице;
\item генерацию оглавлений, списков иллюстраций и таблиц;
\item генерацию предметных указателей;
\item ведение библиографии и др.
\end{enumerate}

\section{Перечень сильных и слабых сторон TeX}
Перечислим основные достоинства TeX:
\begin{enumerate} %перечень
\item LaTeX предоставляет удобные и гибкие средства, чтобы напечатанный вами текст был "совсем как в книге". В частности, указав с помощью простых средств логическую структуру текста, автор может не вникать в детали оформления. Но при необходимости, эти детали могут быть изменены;
\item высокое качество и гибкость верстки абзацев и математических формул;
\item TeX неприхотлив к используемой технике. Например, им можно пользоваться на компьютерах на базе 80286-процессора;
\item TeXовские файлы обладают высокой степенью переносимости. TeX чрезвычайно мобилен и свободно доступен, поэтому система работает практически на всех существующих платформах.
\end{enumerate}

\noindent Однако, есть у TeXа и недостатки:
\begin{enumerate} %перечень
\item TeX работает относительно медленно, занимает много памяти;
\item некоторые пользователи, которые привыкли работать с редакторами наподобие Chiwriter, относят к недостаткам TeXа тот факт, что он не является системой типа WYSIWYG, то есть работа с исходным текстом и просмотр того, как текст будет выглядеть на печати, - разные операции. Но с другой стороны, благодаря этой особенности TeX время на подготовку текста сокращается;
\item переносимость TeXовских текстов снижается, если в них предусмотрен импорт графических файлов;
\item хотя предопределенные макеты имеют множество настраиваемых параметров, создание нового макета документа не очень простое дело и может занять много времени;
\item очень сложно писать неструктурированные и неорганизованные документы.
\end{enumerate}

Не смотря на то, что недостатков получилось больше, преимущества TeX позволяют ему быть популярным среди пользователей, существовать для всех типов компьютеров и применяться для подготовки как одностраничных писем, так и многотомных фолиантов. Многие редакции научных журналов рекомендуют, а иногда и вынуждают готовить статьи в системе LaTeX, чтобы потом, заменив всего лишь одно слово в названии класса печатного документа в преамбуле входного файла, издатель мог придать тексту тот облик, который отличает выбранный журнал.
\section{Инструмент для редактирования TeX - TeXworks}
Для редактирования отчета мной была выбрана свободная среда для работы с TeX-документами, включающая редактор, просмотрщик PDF - TeXworks, обладающая простым интерфейсом. TeXworks является удобным в использовании графическим интерфейсом (GUI) к системе подготовки документов TeX/LaTeX и других. Разработчик данного редактораДжонатан Кью. В комплект TeXworks включен набор шаблонов для создания наиболее часто используемых документов. Редактор включает в себя базовые возможности (характерные для большинства редакторов текста), систему автодополнения команд, запуск вёрстки и возможность добавления сценариев для преобразования текста документа. 
\par
Этот редактор был мной выбран, так как он идеально подходит при необходимости получить PDF файл в результате вёртки (используется pdfTeX). Для предварительного просмотра скомпилированных документов имеется интегрированный просмотрщик PDF (основанный на библиотеке Poppler), поддерживающий синхронизацию с исходным документом.
\chapter{Гитхаб}
\section{Описание сильных сторон git (по сравнению с другими системами контроля версий)}
\section{Перечень полученных навыков и сведений}
\section{git-команды и инструменты, которые удалось освоить}
\section{Ссылки на использованные материалы}
\chapter{Литература}
\chapter{Выполнение индивидуального задания}

\end{document}